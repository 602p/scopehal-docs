\section{Filters}

\subsection{Introduction}

\subsubsection{Key Concepts}

glscopeclient and libscopehal are based on a ``filter graph" architecture internally. The filter graph is a directed
acyclic graph with a set of source nodes (waveforms captured from hardware or loaded from a saved session) and sink
nodes (waveform views, protocol analyzer views, and statistics) connected by edges representing data flow.

A filter is simply an intermediate node in the graph, which takes input from one or more waveform nodes and outputs a
waveform which may be displayed, used as input to other filters, or both. A waveform is a series of data points which
may represent voltages, digital samples, or arbitrarily complex protocol data structures.

As a result, there is no internal distinction between math functions, measurements, and protocol decodes, and it is
possible to chain them arbitrarily. Consider the following example:

\begin{itemize}
\item Two analog waveforms representing serial data and clock are acquired
\item Each analog waveform is thresholded, producing a digital waveform
\item The two digital waveforms are decoded as $I^2C$, producing a series of packets
\item The $I^2C$ packets are decoded as writes to a serial DAC, producing an analog waveform
\item A moving average filter is applied to the analog waveform
\item A measurement filter finds the instantaneous frequency of each cycle of the DAC output
\end{itemize}

In this document we use the term ``filter" consistently to avoid ambiguity.

\subsubsection{Conventions}

Each filter takes one or more inputs (vector inputs), zero or more parameters (scalar inputs), and outputs a
signal (vector output).

If the output signal is a complex-valued type (as opposed to a single scalar, e.g. voltage, at each sample) the
``Output Signal" section will include a table describing how various types of output data are displayed. Printf-style
format codes maybe used for clarity. For example, ``\%02x" means data is formatted as hexadecimal bytes with leading
zeroes.

All filters with complex output use a standardized set of colors to display various types of data fields in a
consistent manner. These colors are currently hard coded in a table but will be made editable in the future
(scopehal-apps:43)

Suggestions on changes to the default colors, or new categories for color coding, are welcome

\begin{tabularx}{16cm}{llX}
\thickhline
\textbf{Color name} & \textbf{Use case} & \textbf{Default Color} \\
\thickhline
Address & Memory addresses & \cellcolor{address}\textcolor{black}{\#ffff00} \\
\thickhline
Checksum Bad & Incorrect CRC/checksum & \cellcolor{checksumbad}\textcolor{white}{\#ff0000} \\
\thickhline
Checksum OK & Valid CRC/checksum & \cellcolor{checksumok}\textcolor{black}{\#00ff00} \\
\thickhline
Control & Miscellaneous control data & \cellcolor{control}\textcolor{white}{\#c000a0} \\
\thickhline
Data & User data & \cellcolor{data}\textcolor{white}{\#336699} \\
\thickhline
Error & Malformed/unreadable data & \cellcolor{error}\textcolor{white}{\#ff0000} \\
\thickhline
Idle & Inter-frame gaps & \cellcolor{idle}\textcolor{white}{\#404040} \\
\thickhline
Preamble & Preamble/sync words & \cellcolor{preamble}\textcolor{white}{\#808080} \\
\thickhline
\end{tabularx}

%%%%%%%%%%%%%%%%%%%%%%%%%%%%%%%%%%%%%%%%%%%%%%%%%%%%%%%%%%%%%%%%%%%%%%%%%%%%%%%%%%%%%%%%%%%%%%%%%%%%%%%%%%%%%%%%%%%%%%%%
\pagebreak
\subsection{8B/10B (IBM)}
\label{proto:8b10b}

Decodes the standard 8b/10b line code used by SGMII, \hyperref[proto:1000basex]{1000base-X}, DisplayPort, JESD204,
PCIe gen 1/2, SATA, USB 3.0, and many other common serial protocols.

\begin{tabularx}{16cm}{llX}
\thickhline
\textbf{Signal name} & \textbf{Type} & \textbf{Description} \\
\thickhline
data & 1-bit digital & Serial 8b/10b data line \\
\thickhline
clk & 1-bit digital & DDR bit clock, typically generated by use of the \hyperref[filter:cdrpll]{Clock Recovery
(PLL)} decode on the input data.\\
\thickhline
\end{tabularx}

\subsubsection{Parameters}

This filter takes no parameters.

\subsubsection{Output Signal}

The 8B/10B filter outputs a time series of 8B/10B sample objects. These consist of a control/data flag and a byte of
data.

\begin{tabularx}{16cm}{lllX}
\thickhline
\textbf{Type} & \textbf{Description} & \textbf{Color} & \textbf{Format} \\
\thickhline
Control & Control codes & \cellcolor{control}\textcolor{white}{Control} & K\%d.\%d \\
\thickhline
Data & Pixel/island data & \cellcolor{data}\textcolor{white}{Data} & D\%d.\%d \\
\thickhline
Error & Malformed data & \cellcolor{error}\textcolor{white}{Error} & ERROR \\
\thickhline
\end{tabularx}

%%%%%%%%%%%%%%%%%%%%%%%%%%%%%%%%%%%%%%%%%%%%%%%%%%%%%%%%%%%%%%%%%%%%%%%%%%%%%%%%%%%%%%%%%%%%%%%%%%%%%%%%%%%%%%%%%%%%%%%%
\pagebreak
\subsection{8B/10B (TMDS)}

Decodes the 8-to-10 Transition Minimized Differential Signalling line code used in DVI and HDMI.

\subsubsection{Inputs}

\begin{tabularx}{16cm}{llX}
\thickhline
\textbf{Signal name} & \textbf{Type} & \textbf{Description} \\
\thickhline
data & 1-bit digital & Serial TMDS data line \\
\thickhline
clk & 1-bit digital & DDR \emph{bit} clock, typically generated by use of the \hyperref[filter:cdrpll]{Clock Recovery
(PLL)} decode on the input data. Note that this is 5x the rate of the HDMI pixel clock signal. \\
\thickhline
\end{tabularx}

\subsubsection{Parameters}

This filter takes no parameters.

\subsubsection{Output Signal}

The TMDS filter outputs a time series of TMDS sample objects. These consist of a type field and a byte of data.

The output of the TMDS decode is commonly fed to the \hyperref[filter:dvi]{DVI} or \hyperref[filter:hdmi]{HDMI}
protocol decoders.

\begin{tabularx}{16cm}{lllX}
\thickhline
\textbf{Type} & \textbf{Description} & \textbf{Color} & \textbf{Format} \\
\thickhline
Control & Control codes (H/V sync) & \cellcolor{control}\textcolor{white}{Control} & CTL\%d \\
\thickhline
Data & Pixel/island data & \cellcolor{data}\textcolor{white}{Data} & \%02x \\
\thickhline
Error & Malformed data & \cellcolor{error}\textcolor{white}{Error} & ERROR \\
\thickhline
Guard band & HDMI data/video guard band & \cellcolor{preamble}\textcolor{white}{Preamble} & GB \\
\thickhline
\end{tabularx}

%%%%%%%%%%%%%%%%%%%%%%%%%%%%%%%%%%%%%%%%%%%%%%%%%%%%%%%%%%%%%%%%%%%%%%%%%%%%%%%%%%%%%%%%%%%%%%%%%%%%%%%%%%%%%%%%%%%%%%%%
\pagebreak
\subsection{AC Couple}

Automatically removes a DC offset from an analog waveform by subtracting the average of all samples from each sample.
Dynamic range is lost compared to doing true AC coupling in the instrument front end, but there are certain situations
where offset removal is necessary in postprocessing.

\subsubsection{Inputs}

\begin{tabularx}{16cm}{llX}
\thickhline
\textbf{Signal name} & \textbf{Type} & \textbf{Description} \\
\thickhline
din & Analog & Input waveform \\
\thickhline
\end{tabularx}

\subsubsection{Parameters}

This filter takes no parameters.

\subsubsection{Output Signal}

This filter outputs an analog waveform with identical sample rate to the input, vertically shifted to center the signal
about zero volts.

%%%%%%%%%%%%%%%%%%%%%%%%%%%%%%%%%%%%%%%%%%%%%%%%%%%%%%%%%%%%%%%%%%%%%%%%%%%%%%%%%%%%%%%%%%%%%%%%%%%%%%%%%%%%%%%%%%%%%%%%
\pagebreak
\subsection{ADL5205}

Decodes SPI data traffic to one half of an ADL5205 variable gain amplifier.

\subsubsection{Inputs}

\begin{tabularx}{16cm}{llX}
\thickhline
\textbf{Signal name} & \textbf{Type} & \textbf{Description} \\
\thickhline
spi & SPI bus & The SPI data bus \\
\thickhline
\end{tabularx}

\subsubsection{Parameters}

This filter takes no parameters.

\subsubsection{Output Signal}

This filter outputs one ADL5205 sample object for each write transaction, formatted as ``write: FA=2 dB, gain=8 dB".

%%%%%%%%%%%%%%%%%%%%%%%%%%%%%%%%%%%%%%%%%%%%%%%%%%%%%%%%%%%%%%%%%%%%%%%%%%%%%%%%%%%%%%%%%%%%%%%%%%%%%%%%%%%%%%%%%%%%%%%%
\pagebreak
\subsection{Base}

Calculates the base (logical zero level) of each cycle in a digital waveform. It is most commonly used as an input to
statistics, to view the average base of the entire waveform.

\subsubsection{Inputs}

\begin{tabularx}{16cm}{llX}
\thickhline
\textbf{Signal name} & \textbf{Type} & \textbf{Description} \\
\thickhline
din & Analog & Input waveform \\
\thickhline
\end{tabularx}

\subsubsection{Parameters}

This filter takes no parameters.

\subsubsection{Output Signal}

This filter outputs an analog waveform with one sample for each group of logical zeroes in the input signal, containing
the average value of the zero level.

%%%%%%%%%%%%%%%%%%%%%%%%%%%%%%%%%%%%%%%%%%%%%%%%%%%%%%%%%%%%%%%%%%%%%%%%%%%%%%%%%%%%%%%%%%%%%%%%%%%%%%%%%%%%%%%%%%%%%%%%
\pagebreak
\subsection{CAN}

%%%%%%%%%%%%%%%%%%%%%%%%%%%%%%%%%%%%%%%%%%%%%%%%%%%%%%%%%%%%%%%%%%%%%%%%%%%%%%%%%%%%%%%%%%%%%%%%%%%%%%%%%%%%%%%%%%%%%%%%
\pagebreak
\subsection{Clock Jitter (TIE)}

%%%%%%%%%%%%%%%%%%%%%%%%%%%%%%%%%%%%%%%%%%%%%%%%%%%%%%%%%%%%%%%%%%%%%%%%%%%%%%%%%%%%%%%%%%%%%%%%%%%%%%%%%%%%%%%%%%%%%%%%
\pagebreak
\subsection{Clock Recovery (PLL)}
\label{filter:cdrpll}

%%%%%%%%%%%%%%%%%%%%%%%%%%%%%%%%%%%%%%%%%%%%%%%%%%%%%%%%%%%%%%%%%%%%%%%%%%%%%%%%%%%%%%%%%%%%%%%%%%%%%%%%%%%%%%%%%%%%%%%%
\pagebreak
\subsection{Clock Recovery (UART)}

%%%%%%%%%%%%%%%%%%%%%%%%%%%%%%%%%%%%%%%%%%%%%%%%%%%%%%%%%%%%%%%%%%%%%%%%%%%%%%%%%%%%%%%%%%%%%%%%%%%%%%%%%%%%%%%%%%%%%%%%
\pagebreak
\subsection{Current Shunt}

Converts a voltage waveform acquired across a known resistance into a current waveform.

%%%%%%%%%%%%%%%%%%%%%%%%%%%%%%%%%%%%%%%%%%%%%%%%%%%%%%%%%%%%%%%%%%%%%%%%%%%%%%%%%%%%%%%%%%%%%%%%%%%%%%%%%%%%%%%%%%%%%%%%
\pagebreak
\subsection{DC Offset}

Adds a constant value to each sample in an analog waveform.

\subsubsection{Inputs}

\begin{tabularx}{16cm}{llX}
\thickhline
\textbf{Signal name} & \textbf{Type} & \textbf{Description} \\
\thickhline
din & Analog & Input waveform \\
\thickhline
\end{tabularx}

\subsubsection{Parameters}

\begin{tabularx}{16cm}{llX}
\thickhline
\textbf{Parameter name} & \textbf{Type} & \textbf{Description} \\
\thickhline
Offset & Float & The offset to apply \\
\thickhline
\end{tabularx}

\subsubsection{Output Signal}

This filter outputs an analog waveform with one sample for each sample in the input, shifted by the requested offset.

%%%%%%%%%%%%%%%%%%%%%%%%%%%%%%%%%%%%%%%%%%%%%%%%%%%%%%%%%%%%%%%%%%%%%%%%%%%%%%%%%%%%%%%%%%%%%%%%%%%%%%%%%%%%%%%%%%%%%%%%
\pagebreak
\subsection{DDR3 Command Bus}

%%%%%%%%%%%%%%%%%%%%%%%%%%%%%%%%%%%%%%%%%%%%%%%%%%%%%%%%%%%%%%%%%%%%%%%%%%%%%%%%%%%%%%%%%%%%%%%%%%%%%%%%%%%%%%%%%%%%%%%%
\pagebreak
\subsection{Deskew}

Moves an analog waveform earlier or later in time to compensate for trigger offsets, probe length mismatch, etc.
It is generally preferable to deskew using the skew adjustment on the channel during acquisition; this filter is
provided for correction in postprocessing.

\subsubsection{Inputs}

\begin{tabularx}{16cm}{llX}
\thickhline
\textbf{Signal name} & \textbf{Type} & \textbf{Description} \\
\thickhline
din & Analog & Input waveform \\
\thickhline
\end{tabularx}

\subsubsection{Parameters}

\begin{tabularx}{16cm}{llX}
\thickhline
\textbf{Parameter name} & \textbf{Type} & \textbf{Description} \\
\thickhline
Skew & Float & Time offset to shift the waveform\\
\thickhline
\end{tabularx}

\subsubsection{Output Signal}

This filter outputs an analog waveform with one sample for each sample in the input, phase shifted by the requested
offset.

%%%%%%%%%%%%%%%%%%%%%%%%%%%%%%%%%%%%%%%%%%%%%%%%%%%%%%%%%%%%%%%%%%%%%%%%%%%%%%%%%%%%%%%%%%%%%%%%%%%%%%%%%%%%%%%%%%%%%%%%
\pagebreak
\subsection{DRAM Trcd}

%%%%%%%%%%%%%%%%%%%%%%%%%%%%%%%%%%%%%%%%%%%%%%%%%%%%%%%%%%%%%%%%%%%%%%%%%%%%%%%%%%%%%%%%%%%%%%%%%%%%%%%%%%%%%%%%%%%%%%%%
\pagebreak
\subsection{DRAM Trfc}

%%%%%%%%%%%%%%%%%%%%%%%%%%%%%%%%%%%%%%%%%%%%%%%%%%%%%%%%%%%%%%%%%%%%%%%%%%%%%%%%%%%%%%%%%%%%%%%%%%%%%%%%%%%%%%%%%%%%%%%%
\pagebreak
\subsection{DVI}
\label{filter:dvi}

%%%%%%%%%%%%%%%%%%%%%%%%%%%%%%%%%%%%%%%%%%%%%%%%%%%%%%%%%%%%%%%%%%%%%%%%%%%%%%%%%%%%%%%%%%%%%%%%%%%%%%%%%%%%%%%%%%%%%%%%
\pagebreak
\subsection{Ethernet - 10baseT}

%%%%%%%%%%%%%%%%%%%%%%%%%%%%%%%%%%%%%%%%%%%%%%%%%%%%%%%%%%%%%%%%%%%%%%%%%%%%%%%%%%%%%%%%%%%%%%%%%%%%%%%%%%%%%%%%%%%%%%%%
\pagebreak
\subsection{Ethernet - 100baseTX}

%%%%%%%%%%%%%%%%%%%%%%%%%%%%%%%%%%%%%%%%%%%%%%%%%%%%%%%%%%%%%%%%%%%%%%%%%%%%%%%%%%%%%%%%%%%%%%%%%%%%%%%%%%%%%%%%%%%%%%%%
\pagebreak
\subsection{Ethernet - 1000baseX}
\label{proto:1000basex}

Decodes the 1000base-X Ethernet PCS as specified in IEEE 802.3 clause 36.

\begin{tabularx}{16cm}{llX}
\thickhline
\textbf{Signal name} & \textbf{Type} & \textbf{Description} \\
\thickhline
data & 8b/10b & Output of \hyperref[proto:8b10b]{8b/10b protocol decode}\\
\thickhline
\end{tabularx}

\subsubsection{Parameters}

This filter takes no parameters.

\subsubsection{Output Signal}

The 1000base-X filter outputs a series of Ethernet frame segment objects.

\begin{tabularx}{16cm}{lllX}
\thickhline
\textbf{Type} & \textbf{Description} & \textbf{Color} & \textbf{Format} \\
\thickhline
Preamble & Preamble & \cellcolor{preamble}\textcolor{white}{Preamble} & PREAMBLE \\
\thickhline
Preamble & Start of frame delimiter & \cellcolor{preamble}\textcolor{white}{Preamble} & SFD \\
\thickhline
Address & Src/dest MAC & \cellcolor{address}\textcolor{black}{Address} & 02:00:11:22:33:44 \\
\thickhline
Control & Ethertype & \cellcolor{control}\textcolor{white}{Control} & Type: IPv4 \newline Type: 0xbeef \\
\thickhline
Control & VLAN tag & \cellcolor{control}\textcolor{white}{Control} & VLAN 10, PCP 0 \\
\thickhline
Data & Frame data & \cellcolor{data}\textcolor{white}{Data} & a5 \\
\thickhline
Checksum OK & Valid FCS & \cellcolor{checksumok}\textcolor{black}{Checksum OK} & CRC: 0xdeadbeef \\
\thickhline
Checksum Bad & Invalid FCS & \cellcolor{checksumbad}\textcolor{white}{Checksum Bad} & CRC: 0xbaadc0de \\
\thickhline
Error & Malformed data & \cellcolor{error}\textcolor{white}{Error} & ERROR \\
\thickhline
\end{tabularx}

TODO: Document protocol analyzer output

%%%%%%%%%%%%%%%%%%%%%%%%%%%%%%%%%%%%%%%%%%%%%%%%%%%%%%%%%%%%%%%%%%%%%%%%%%%%%%%%%%%%%%%%%%%%%%%%%%%%%%%%%%%%%%%%%%%%%%%%
\pagebreak
\subsection{Ethernet - GMII}

%%%%%%%%%%%%%%%%%%%%%%%%%%%%%%%%%%%%%%%%%%%%%%%%%%%%%%%%%%%%%%%%%%%%%%%%%%%%%%%%%%%%%%%%%%%%%%%%%%%%%%%%%%%%%%%%%%%%%%%%
\pagebreak
\subsection{Ethernet - RGMII}

%%%%%%%%%%%%%%%%%%%%%%%%%%%%%%%%%%%%%%%%%%%%%%%%%%%%%%%%%%%%%%%%%%%%%%%%%%%%%%%%%%%%%%%%%%%%%%%%%%%%%%%%%%%%%%%%%%%%%%%%
\pagebreak
\subsection{Ethernet Autonegotiation}

%%%%%%%%%%%%%%%%%%%%%%%%%%%%%%%%%%%%%%%%%%%%%%%%%%%%%%%%%%%%%%%%%%%%%%%%%%%%%%%%%%%%%%%%%%%%%%%%%%%%%%%%%%%%%%%%%%%%%%%%
\pagebreak
\subsection{Eye Bit Rate}

%%%%%%%%%%%%%%%%%%%%%%%%%%%%%%%%%%%%%%%%%%%%%%%%%%%%%%%%%%%%%%%%%%%%%%%%%%%%%%%%%%%%%%%%%%%%%%%%%%%%%%%%%%%%%%%%%%%%%%%%
\pagebreak
\subsection{Eye Height}

%%%%%%%%%%%%%%%%%%%%%%%%%%%%%%%%%%%%%%%%%%%%%%%%%%%%%%%%%%%%%%%%%%%%%%%%%%%%%%%%%%%%%%%%%%%%%%%%%%%%%%%%%%%%%%%%%%%%%%%%
\pagebreak
\subsection{Eye P-P Jitter}

%%%%%%%%%%%%%%%%%%%%%%%%%%%%%%%%%%%%%%%%%%%%%%%%%%%%%%%%%%%%%%%%%%%%%%%%%%%%%%%%%%%%%%%%%%%%%%%%%%%%%%%%%%%%%%%%%%%%%%%%
\pagebreak
\subsection{Eye Pattern}

%%%%%%%%%%%%%%%%%%%%%%%%%%%%%%%%%%%%%%%%%%%%%%%%%%%%%%%%%%%%%%%%%%%%%%%%%%%%%%%%%%%%%%%%%%%%%%%%%%%%%%%%%%%%%%%%%%%%%%%%
\pagebreak
\subsection{Eye Period}

%%%%%%%%%%%%%%%%%%%%%%%%%%%%%%%%%%%%%%%%%%%%%%%%%%%%%%%%%%%%%%%%%%%%%%%%%%%%%%%%%%%%%%%%%%%%%%%%%%%%%%%%%%%%%%%%%%%%%%%%
\pagebreak
\subsection{Eye Width}

%%%%%%%%%%%%%%%%%%%%%%%%%%%%%%%%%%%%%%%%%%%%%%%%%%%%%%%%%%%%%%%%%%%%%%%%%%%%%%%%%%%%%%%%%%%%%%%%%%%%%%%%%%%%%%%%%%%%%%%%
\pagebreak
\subsection{Fall}

%%%%%%%%%%%%%%%%%%%%%%%%%%%%%%%%%%%%%%%%%%%%%%%%%%%%%%%%%%%%%%%%%%%%%%%%%%%%%%%%%%%%%%%%%%%%%%%%%%%%%%%%%%%%%%%%%%%%%%%%
\pagebreak
\subsection{FFT}

%%%%%%%%%%%%%%%%%%%%%%%%%%%%%%%%%%%%%%%%%%%%%%%%%%%%%%%%%%%%%%%%%%%%%%%%%%%%%%%%%%%%%%%%%%%%%%%%%%%%%%%%%%%%%%%%%%%%%%%%
\pagebreak
\subsection{Frequency}

%%%%%%%%%%%%%%%%%%%%%%%%%%%%%%%%%%%%%%%%%%%%%%%%%%%%%%%%%%%%%%%%%%%%%%%%%%%%%%%%%%%%%%%%%%%%%%%%%%%%%%%%%%%%%%%%%%%%%%%%
\pagebreak
\subsection{Horizontal Bathtub}

%%%%%%%%%%%%%%%%%%%%%%%%%%%%%%%%%%%%%%%%%%%%%%%%%%%%%%%%%%%%%%%%%%%%%%%%%%%%%%%%%%%%%%%%%%%%%%%%%%%%%%%%%%%%%%%%%%%%%%%%
\pagebreak
\subsection{HDMI}
\label{filter:hdmi}

%%%%%%%%%%%%%%%%%%%%%%%%%%%%%%%%%%%%%%%%%%%%%%%%%%%%%%%%%%%%%%%%%%%%%%%%%%%%%%%%%%%%%%%%%%%%%%%%%%%%%%%%%%%%%%%%%%%%%%%%
\pagebreak
\subsection{$I^2C$}

%%%%%%%%%%%%%%%%%%%%%%%%%%%%%%%%%%%%%%%%%%%%%%%%%%%%%%%%%%%%%%%%%%%%%%%%%%%%%%%%%%%%%%%%%%%%%%%%%%%%%%%%%%%%%%%%%%%%%%%%
\pagebreak
\subsection{JTAG}

%%%%%%%%%%%%%%%%%%%%%%%%%%%%%%%%%%%%%%%%%%%%%%%%%%%%%%%%%%%%%%%%%%%%%%%%%%%%%%%%%%%%%%%%%%%%%%%%%%%%%%%%%%%%%%%%%%%%%%%%
\pagebreak
\subsection{MDIO}

%%%%%%%%%%%%%%%%%%%%%%%%%%%%%%%%%%%%%%%%%%%%%%%%%%%%%%%%%%%%%%%%%%%%%%%%%%%%%%%%%%%%%%%%%%%%%%%%%%%%%%%%%%%%%%%%%%%%%%%%
\pagebreak
\subsection{Moving Average}

%%%%%%%%%%%%%%%%%%%%%%%%%%%%%%%%%%%%%%%%%%%%%%%%%%%%%%%%%%%%%%%%%%%%%%%%%%%%%%%%%%%%%%%%%%%%%%%%%%%%%%%%%%%%%%%%%%%%%%%%
\pagebreak
\subsection{Multiply}

Multiplies one waveform by another. No resampling is performed; both inputs must have identical sample rates.

Unit conversions are performed, for example the product of a voltage and current waveform is a power waveform.

%%%%%%%%%%%%%%%%%%%%%%%%%%%%%%%%%%%%%%%%%%%%%%%%%%%%%%%%%%%%%%%%%%%%%%%%%%%%%%%%%%%%%%%%%%%%%%%%%%%%%%%%%%%%%%%%%%%%%%%%
\pagebreak
\subsection{Overshoot}

%%%%%%%%%%%%%%%%%%%%%%%%%%%%%%%%%%%%%%%%%%%%%%%%%%%%%%%%%%%%%%%%%%%%%%%%%%%%%%%%%%%%%%%%%%%%%%%%%%%%%%%%%%%%%%%%%%%%%%%%
\pagebreak
\subsection{Parallel Bus}

%%%%%%%%%%%%%%%%%%%%%%%%%%%%%%%%%%%%%%%%%%%%%%%%%%%%%%%%%%%%%%%%%%%%%%%%%%%%%%%%%%%%%%%%%%%%%%%%%%%%%%%%%%%%%%%%%%%%%%%%
\pagebreak
\subsection{Peak-to-Peak}

%%%%%%%%%%%%%%%%%%%%%%%%%%%%%%%%%%%%%%%%%%%%%%%%%%%%%%%%%%%%%%%%%%%%%%%%%%%%%%%%%%%%%%%%%%%%%%%%%%%%%%%%%%%%%%%%%%%%%%%%
\pagebreak
\subsection{Period}

%%%%%%%%%%%%%%%%%%%%%%%%%%%%%%%%%%%%%%%%%%%%%%%%%%%%%%%%%%%%%%%%%%%%%%%%%%%%%%%%%%%%%%%%%%%%%%%%%%%%%%%%%%%%%%%%%%%%%%%%
\pagebreak
\subsection{Rise}

%%%%%%%%%%%%%%%%%%%%%%%%%%%%%%%%%%%%%%%%%%%%%%%%%%%%%%%%%%%%%%%%%%%%%%%%%%%%%%%%%%%%%%%%%%%%%%%%%%%%%%%%%%%%%%%%%%%%%%%%
\pagebreak
\subsection{SPI}

%%%%%%%%%%%%%%%%%%%%%%%%%%%%%%%%%%%%%%%%%%%%%%%%%%%%%%%%%%%%%%%%%%%%%%%%%%%%%%%%%%%%%%%%%%%%%%%%%%%%%%%%%%%%%%%%%%%%%%%%
\pagebreak
\subsection{Subtract}

Subtracts one waveform from another. No resampling is performed; both inputs must have identical sample rates.

\subsubsection{Inputs}

\begin{tabularx}{16cm}{llX}
\thickhline
\textbf{Signal name} & \textbf{Type} & \textbf{Description} \\
\thickhline
IN+ & Analog & Positive input waveform \\
\thickhline
IN- & Analog & Negative input waveform \\
\thickhline
\end{tabularx}

\subsubsection{Parameters}

This filter takes no parameters.

\subsubsection{Output Signal}

This filter outputs an analog waveform with one sample for each sample in the input, containing the difference of the
two input waveforms.

%%%%%%%%%%%%%%%%%%%%%%%%%%%%%%%%%%%%%%%%%%%%%%%%%%%%%%%%%%%%%%%%%%%%%%%%%%%%%%%%%%%%%%%%%%%%%%%%%%%%%%%%%%%%%%%%%%%%%%%%
\pagebreak
\subsection{Threshold}

Converts an analog waveform to digital by thresholding at a constant level (no hysteresis).

\subsubsection{Inputs}

\begin{tabularx}{16cm}{llX}
\thickhline
\textbf{Signal name} & \textbf{Type} & \textbf{Description} \\
\thickhline
din & Analog & Input waveform \\
\thickhline
\end{tabularx}

\subsubsection{Parameters}

\begin{tabularx}{16cm}{llX}
\thickhline
\textbf{Parameter name} & \textbf{Type} & \textbf{Description} \\
\thickhline
Threshold & Float & Decision threshold \\
\thickhline
\end{tabularx}

\subsubsection{Output Signal}

This filter outputs an digital waveform with one sample for each sample in the input, which is true if the
corresponding input sample is above the threshold and false if less than or equal.

%%%%%%%%%%%%%%%%%%%%%%%%%%%%%%%%%%%%%%%%%%%%%%%%%%%%%%%%%%%%%%%%%%%%%%%%%%%%%%%%%%%%%%%%%%%%%%%%%%%%%%%%%%%%%%%%%%%%%%%%
\pagebreak
\subsection{Top}

Calculates the top (logical one level) of each cycle in a digital waveform. It is most commonly used as an input to
statistics, to view the average top of the entire waveform.

\subsubsection{Inputs}

\begin{tabularx}{16cm}{llX}
\thickhline
\textbf{Signal name} & \textbf{Type} & \textbf{Description} \\
\thickhline
din & Analog & Input waveform \\
\thickhline
\end{tabularx}

\subsubsection{Parameters}

This filter takes no parameters.

\subsubsection{Output Signal}

This filter outputs an analog waveform with one sample for each group of logical ones in the input signal, containing
the average value of the one level.

%%%%%%%%%%%%%%%%%%%%%%%%%%%%%%%%%%%%%%%%%%%%%%%%%%%%%%%%%%%%%%%%%%%%%%%%%%%%%%%%%%%%%%%%%%%%%%%%%%%%%%%%%%%%%%%%%%%%%%%%
\pagebreak
\subsection{UART}

%%%%%%%%%%%%%%%%%%%%%%%%%%%%%%%%%%%%%%%%%%%%%%%%%%%%%%%%%%%%%%%%%%%%%%%%%%%%%%%%%%%%%%%%%%%%%%%%%%%%%%%%%%%%%%%%%%%%%%%%
\pagebreak
\subsection{USB 1.0 / 2.x Activity}

%%%%%%%%%%%%%%%%%%%%%%%%%%%%%%%%%%%%%%%%%%%%%%%%%%%%%%%%%%%%%%%%%%%%%%%%%%%%%%%%%%%%%%%%%%%%%%%%%%%%%%%%%%%%%%%%%%%%%%%%
\pagebreak
\subsection{USB 1.0 / 2.x Packet}

%%%%%%%%%%%%%%%%%%%%%%%%%%%%%%%%%%%%%%%%%%%%%%%%%%%%%%%%%%%%%%%%%%%%%%%%%%%%%%%%%%%%%%%%%%%%%%%%%%%%%%%%%%%%%%%%%%%%%%%%
\pagebreak
\subsection{USB 1.0 / 2.x PCS}

%%%%%%%%%%%%%%%%%%%%%%%%%%%%%%%%%%%%%%%%%%%%%%%%%%%%%%%%%%%%%%%%%%%%%%%%%%%%%%%%%%%%%%%%%%%%%%%%%%%%%%%%%%%%%%%%%%%%%%%%
\pagebreak
\subsection{USB 1.0 / 2.x PMA}

%%%%%%%%%%%%%%%%%%%%%%%%%%%%%%%%%%%%%%%%%%%%%%%%%%%%%%%%%%%%%%%%%%%%%%%%%%%%%%%%%%%%%%%%%%%%%%%%%%%%%%%%%%%%%%%%%%%%%%%%
\pagebreak
\subsection{Undershoot}

%%%%%%%%%%%%%%%%%%%%%%%%%%%%%%%%%%%%%%%%%%%%%%%%%%%%%%%%%%%%%%%%%%%%%%%%%%%%%%%%%%%%%%%%%%%%%%%%%%%%%%%%%%%%%%%%%%%%%%%%
\pagebreak
\subsection{Upsample}

%%%%%%%%%%%%%%%%%%%%%%%%%%%%%%%%%%%%%%%%%%%%%%%%%%%%%%%%%%%%%%%%%%%%%%%%%%%%%%%%%%%%%%%%%%%%%%%%%%%%%%%%%%%%%%%%%%%%%%%%
\pagebreak
\subsection{Waterfall}
