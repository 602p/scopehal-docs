\documentclass[11pt]{article}
\usepackage[T1]{fontenc}
\usepackage[margin=1in,top=0.6in,bottom=0.6in]{geometry}
\usepackage[bookmarks,colorlinks=true,linkcolor=blue,urlcolor=blue]{hyperref}
\usepackage{url}
\usepackage{tabularx}
\usepackage{graphicx}
\usepackage{placeins}
\usepackage{paralist}
\usepackage{makecell}
\usepackage{tikz}
\usepackage[osf]{libertine}
\usepackage{zi4}
\usepackage[libertine,cmbraces]{newtxmath}

% configuration of source code examples
\usepackage{listings}
\lstset{language=c++}
\lstset{numbers=left}
\lstset{xleftmargin=2em}
\lstset{framexleftmargin=2em}
\lstset{belowskip=0em}
\lstset{belowcaptionskip=0em}
\lstset{tabsize=4}
\lstset{frame=single}
\lstset{breaklines=true}
\lstset{showspaces=false}
\lstset{showstringspaces=false}
\lstset{showtabs=false}
\lstset{breakatwhitespace=false}
\lstset{basicstyle=\small\ttfamily}

% table lines
\newcommand{\thinhline}{\Xhline{1\arrayrulewidth}}
\newcommand{\thickhline}{\Xhline{2.5\arrayrulewidth}}

\setcounter{tocdepth}{2}

\begin{document}

\title{glscopeclient Operator Manual}
\author{Andrew Zonenberg\\
azonenberg@drawersteak.com}
\date{\today}

\maketitle
\begin{abstract} \normalsize
This document is the user manual for glscopeclient, a user interface and signal analysis tool for oscilloscopes and
logic analyzers. As of this writing, glscopeclient is under active development but has not had a formal v0.1 release
and should be considered alpha quality.
\end{abstract}
\thispagestyle{empty}

\pagebreak

\tableofcontents

\pagebreak
\section{Revision History}
\begin{itemize}
\item \today: [in progress] Initial draft
\end{itemize}

\pagebreak
\section{Getting Started}

\subsection{Supported Hardware}

glscopeclient uses the libscopehal library to communicate with oscilloscopes, so any libscopehal-compatible hardware
should work with glscopeclient.

\begin{tabularx}{16cm}{lllllX}
\thickhline
Vendor & Device families & Driver & Tested on & Status \\
\thickhline
R\&S & RTM3000 & rs\_lan & (FIXME) & \makecell{Read-only mostly works\\can't change most settings} \\
\thickhline
Rigol & DS1000Z & rigol\_lan & DS1054Z & \makecell{Read-only mostly works\\can't change most settings} \\
\thickhline
Siglent & FIXME & lecroy\_lan & Untested & FIXME \\
\thickhline
Siglent & FIXME & lecroy\_vicp & Untested & FIXME \\
\thickhline
\makecell{Teledyne \\ LeCroy} & MAUI based & lecroy\_vicp & \makecell{WaveRunner 8104 \\ HDO9204} & Fully usable \\
\thickhline
\makecell{Teledyne \\ LeCroy} & T3DSO & lecroy\_lan & Untested & WIP \\
\thickhline
\end{tabularx}

\subsection{Compilation}

\begin{enumerate}

\item Install dependencies. On Debian/Ubuntu:
\begin{lstlisting}[language=sh]
sudo apt install build-essential cmake pkg-config libglm-dev \
	libgtkmm-3.0-dev libsigc++-2.0-dev
\end{lstlisting}

\item Install FFTS library
\begin{lstlisting}[language=sh]
git clone https://github.com/anthonix/ffts.git
cd ffts
mkdir build
cd build
cmake ../
sudo make install
\end{lstlisting}

\item Build scopehal and scopehal-apps
\begin{lstlisting}[language=sh]
git clone https://github.com/azonenberg/scopehal-cmake.git --recurse-submodules
cd scopehal-cmake
mkdir build
cd build
cmake ../
make
\end{lstlisting}

\item Install scopehal and scopehal-apps: right now, you don't. As of now, glscopeclient is intended to be run from the
glscopeclient binary directory (build/src/glscopeclient). Anybody want to contribute and set up a proper install
process?

\end{enumerate}

\subsection{Running glscopeclient}

There is not yet a proper GUI startup dialog for discovering and connecting to instruments. For the moment, you must
specify the instrument(s) you plan to connect to on the command line.

\begin{lstlisting}[language=sh]
./glscopeclient --debug \
	mylecroy:lecroy_vicp:myscope.example.com:1234 \
	myrigol:rigol_lan:rigol.example.com
\end{lstlisting}

The --debug argument may be omitted or replaced with any other liblogtools argument for controlling console debug
verbosity (--quiet, --verbose, --debug, --trace, etc). If you're using glscopeclient at its current level of maturity
you're probably a developer, so we suggest --debug.

Each instrument is described by a ``connection string" containing three colon-separated fields.

\begin{itemize}
\item Nickname. This can be any text string not containing spaces or colons. If you have only one instrument it's
largely ignored, but when multiple instruments are present channel names in the UI are prefixed with the nickname to
avoid ambiguity.
\item Driver name. This is a string identifying the command protocol and interface the scope uses. Note that not all
scopes from the same vendor will use the same command set or driver!
\item Arguments for the driver identifying the device to connect to, separated by colons. This varies by driver but is
typically a hostname:port combination, TTY device path, or similar.
\end{itemize}

\pagebreak
\section{Oscilloscope Drivers}

\subsection{Bar}
TODO

\pagebreak
\section{User Interface Overview}

\end{document}
