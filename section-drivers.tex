\chapter{Oscilloscope Drivers}
\label{sec:drivers}

\section{Agilent}

Agilent devices support a similar similar SCPI command set across most device families.

Please see the table below for details of current hardware support:

\begin{tabularx}{16cm}{llX}
\thickhline
\textbf{Device Family} & \textbf{Driver} & \textbf{Notes} \\
\thickhline
DSO5000 series & agilent & Not recently tested, but should work.\\
\thickhline
DSO6000 \& MSO6000 series & agilent &  Working. No support for digital channels yet.\\
\thickhline
DSO7000 \& MSO7000 series & agilent &  Untested, but should work. No support for digital channels yet.\\
\thickhline
MSOX-2000 series & agilent &  \\
\thickhline
MSOX-3000 series & agilent &  \\
\thickhline
\end{tabularx}

\subsection{agilent}

This driver has been tested on an MSO6034A.

\section{Antikernel Labs}

\begin{tabularx}{16cm}{llX}
\thickhline
\textbf{Device Family} & \textbf{Driver} & \textbf{Notes} \\
\thickhline
Internal Logic Analyzer IP & akila & \\
\thickhline
BLONDEL Oscilloscope Prototype & aklabs & \\
\thickhline
\end{tabularx}

\subsection{akila}

This driver uses a raw binary protocol, not SCPI.

Under-development internal logic analyzer analyzer core for FPGA design debug. The ILA uses a UART interface to a host
system. Since there's no UART support in scopehal yet, socat must be used to bridge the UART to a TCP socket using
the ``lan" transport.

\subsection{aklabs}

This driver uses two TCP sockets. Port 5025 is used for SCPI control plane traffic, and port 50101 is used for waveform
data using a raw binary protocol.

\section{Demo}

The ``demo" driver is a simulation-only driver for development and training purposes, and does not connect to real
hardware.

It ignores any transport provided, and is normally used with the ``null" transport.

The demo instrument is intended to illustrate the usage of glscopeclient for various types of analysis and to aid in
automated testing on computers which do not have a connection to a real oscilloscope, and is not intended to accurately
model the response or characteristics of real world scope frontends or signals.

It supports memory depths of 10K, 100K, 1M, and 10M points per waveform at rates of 1, 5, 10, 25, 50, and 100 Gsps.
Four test signals are provided, each with 10 mV of Gaussian noise added. As of now, there is no way to configure the
simulated signal generator.

Test signals:
\begin{itemize}
\item 1.000 GHz tone
\item 1.000 GHz tone mixed with a second tone, which sweeps from 1.100 to 1.500 GHz
\item 10.3125 Gbps PRBS-31
\item 1.25 Gbps repeating two 8B/10B symbols (K28.5 D16.2)
\end{itemize}

\begin{tabularx}{16cm}{lllX}
\thickhline
\textbf{Device Family} & \textbf{Driver} & \textbf{Transport} & \textbf{Notes} \\
\thickhline
Simulator & demo & null & \\
\thickhline
\end{tabularx}

\section{Enjoy Digital}
TODO (scopehal:79)

\section{Hantek}
TODO (scopehal:26)

\section{Keysight}

Keysight devices support a similar similar SCPI command set across most device families. Many Keysight devices were previously sold under the Agilent brand, so they are supported by the ``agilent" driver.

Please see the table below for details of current hardware support:

\begin{tabularx}{16cm}{llX}
\thickhline
\textbf{Device Family} & \textbf{Driver} & \textbf{Notes} \\
\thickhline
MSOX-2000 series & agilent &  \\
\thickhline
MSOX-3000 series & agilent &  \\
\thickhline
\end{tabularx}

\subsection{agilent}

\section{Pico Technologies}
TODO (scopehal:15)

\section{Rigol}

Rigol oscilloscopes have subtle differences in SCPI command set, but this is implemented with quirks handling in the
driver rather than needing different drivers for each scope family.

\begin{tabularx}{16cm}{llX}
\thickhline
\textbf{Device Family} & \textbf{Driver} & \textbf{Notes} \\
\thickhline
DS1100D/E & rigol & \\
\thickhline
DS1000Z & rigol & \\
\thickhline
MSO5000 & rigol & \\
\thickhline
\end{tabularx}

\subsection{rigol}

This driver has been primarily tested on a MSO5000 series scope.

\section{Rohde \& Schwarz}

There is partial support for RTM3000 (and possibly others, untested) however it appears to have bitrotted.

TODO (scopehal:59)

\section{Saleae}
TODO (scopehal:16)

\section{Siglent}

Many recent Siglent oscilloscopes are developed in partnership with Teledyne LeCroy (Siglent-designed hardware running
Teledyne LeCroy firmware) and are sold under both brands. As a result, there is some crossover in driver support. \\

The current Siglent driver appears to have bitrotted due to divergence between the ``lecroy" and ``siglent" driver
codebases. Contributions to fix Siglent support are welcome!

\begin{tabularx}{16cm}{lllX}
\thickhline
\textbf{Device Family} & \textbf{Driver} & \textbf{Transport} & \textbf{Notes} \\
\thickhline
SDS5000X series & siglent & lxi, lan, usbtmc & Untested \\
\thickhline
SDS2000X Plus series & siglent & lxi, lan, usbtmc & Untested \\
\thickhline
SDS2000X-E series & siglent & lxi, lan, usbtmc & Untested \\
\thickhline
SDS2000X series & siglent & lxi, usbtmc & Base functionality present, lxi and usbtmc tested on SDS2304X. \\
\thickhline
SDS1000X series & siglent & lxi, usbtmc & Untested \\
\thickhline
SDS1000X+ series & siglent & lxi, usbtmc & Untested \\
\thickhline
SDS1000X-E series & siglent & lxi, lan, usbtmc & Base functionality present, lan transport tested on SDS1204X-E.\\
\thickhline
SDS1000CFL series & siglent & usbtmc, ? & Untested. Documentation contradiction about support over Ethernet. \\
\thickhline
SDS1000DL+ series & siglent & lxi, usbtmc & Untested \\
\thickhline
SDS1000CML+ series & siglent & lxi, usbtmc & Untested \\
\thickhline
SDS3000X series & lecroy & vicp & Should be same as WaveSurfer 3000 \\
\end{tabularx}


Unlike TCP/IP sockets (``lan"), VXI-11 (``lxi") is a synchronous protocol that  does not support
queueing multiple transactions. When an oscilloscope supports both the lan and the lxi transport,
chances are that the lan transport will have a better performance than the lxi transport.

\section{Teledyne LeCroy / LeCroy}

Teledyne LeCroy (and older LeCroy) devices use the same driver, but two different transports for LAN connections.

While all Teledyne LeCroy / LeCroy devices use almost identical SCPI command sets, Windows based devices running
XStream or MAUI use a custom framing protocol (``vicp") around the SCPI data while the lower end RTOS based devices use
raw SCPI over TCP (``lan"). Some of the lower end devices also require use of the Siglent driver as they are Siglent
OEM designs rebranded by Teledyne LeCroy and have some quirks in the firmware which require workarounds.

Please see the table below for details on which configuration to use with  your hardware.

\begin{tabularx}{16cm}{lllX}
\thickhline
\textbf{Device Family} & \textbf{Driver} & \textbf{Transport} & \textbf{Notes} \\
\thickhline
DDA & lecroy & vicp & Tested on DDA5000A series \\
\thickhline
HDO & lecroy & vicp & Tested on HDO9000 series \\
\thickhline
LabMaster & lecroy & vicp & Untested, but should work\\
\thickhline
MDA & lecroy & vicp & Untested, but should work\\
\thickhline
SDA & lecroy & vicp & Untested, but should work\\
\thickhline
T3DSO & siglent & ??? & Untested\\
\thickhline
WaveAce & ??? & ??? & Untested\\
\thickhline
WaveJet & ??? & ??? & Untested\\
\thickhline
WaveMaster & lecroy & vicp & Untested, but should work \\
\thickhline
WaveRunner & lecroy & vicp & Tested on WaveRunner Xi and 8000 series\\
\thickhline
WaveSurfer & lecroy & vicp & Tested on WaveSurfer 3000 series \\
\thickhline
\end{tabularx}

\subsection{lecroy}

This is the primary driver for MAUI based Teledyne LeCroy / LeCroy devices.

This driver has been tested on a wide range of Teledyne LeCroy / LeCroy hardware. It should be compatible with any
Teledyne LeCroy or LeCroy oscilloscope running Windows XP or newer and the MAUI or XStream software.

\section{Tektronix}

This driver is being primarily developed on a MSO64. It supports raw SCPI over TCP using the ``lan" transport,
using port 4000 by default rather than the IANA standard of 5025, as well as LXI VXI-11. It may also work via USBTMC,
but this have never been tested.

Note that the default settings for raw SCPI access on the MSO6 series use a full terminal emulator rather than raw SCPI
commands. To remove the prompts and help text, go to Utility | I/O, then under the Socket Server panel select protocol
``None" rather than the default of ``Terminal". Alternatively, use the LXI transport rather than raw SCPI.

\begin{tabularx}{16cm}{lllX}
\thickhline
\textbf{Device Family} & \textbf{Driver} & \textbf{Transport} & \textbf{Notes} \\
\thickhline
MSO5 series & tektronix & lan, lxi &  \\
\thickhline
MSO6 series & tektronix & lan, lxi &  \\
\thickhline
\end{tabularx}

\section{Xilinx}
TODO (scopehal:40)
