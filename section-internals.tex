\chapter{Internals}

\section{Introduction}

This chapter provides a high level overview of libscopehal and glscopeclient internals. It is intended for developers
to gain an understanding of the overall project architecture and how key pieces fit together, but is not a substitute
for the low level API documentation (Doxygen).

Many of the entities described below use a dynamic discovery / registration system. This allows all such classes to be
enumerated (and associated with human-readable names), and allows for objects of any registered type - including those
provided by plugins - to be created at run time by a factory method given the human-readable class name.

\section{Instruments}
\label{sec:instruments}

An instrument is an instance of a class derived from \codestyle{Instrument}, which represents an arbitrary piece of
laboratory equipment. As of this writing, an instrument may be an oscilloscope, multimeter, power supply, baseband
signal generator, or RF signal generator - or an arbitrary combination of these (for example an oscilloscope with
integrated function generator is both an oscilloscope and baseband signal generator).

The type of an instrument is defined by a bit field and may be queried by calling \codestyle{GetInstrumentTypes()}. Do
\emph{not} rely on C++ RTTI to determine the type of an instrument, for example it is incorrect to
\codestyle{dynamic\_cast} a \codestyle{Instrument*} pointer to \codestyle{Oscilloscope*} to check if the instrument is
an oscilloscope. This is because the C++ type of an object is fixed when the driver class is compiled, and the driver
may be used with many different instruments with various sets of software and hardware options. In other words, the
fact that a given driver supports \emph{some} device that contains multimeter functionality does not in any way imply
that the \emph{particular} device you are talking to is a multimeter.

The Instrument class provides no functionality other than describing the device (querying make/model/serial number,
assigning display nicknames, and querying feature set). To do any useful work, the object is normally casted to a
derived type to gain access to that device class's API.

\section{SCPI Devices}
\label{sec:scpidevices}

A SCPI device is an instance of a class derived from \codestyle{SCPIDevice}, which represents a device which speaks
some variant of SCPI. The vast majority of instrument driver classes derive from both \codestyle{SCPIDevice} and one or
more \codestyle{Instrument} derived classes.

A SCPI device object uses a \hyperref[sec:transports]{transport} to communicate with the associated instrument, which
avoids the need for the driver class to concern itself with the specifics of how the SCPI commands are transferred to
the device.

\section{Transports}
\label{sec:transports}

A transport is an instance of a class derived from \codestyle{SCPITransport}, which provides a means of sending SCPI
commands and/or raw byte string data to or from a physical instrument.

Most transports use a single stream in the underlying protocol layer (such as a single TCP socket) to transport both
control plane content (SCPI commands) and data plane content (waveform data), however some specialized protocols have
multiple physical streams (for example the \codestyle{SCPITwinLanTransport} transport). For these instruments, the
command/reply APIs and raw data APIs may not go to the same place.

All transports must be registered in order to be used by glscopeclient. To register a transport class, add the macro
\codestyle{TRANSPORT\_INITPROC(FooTransport)} to your class declaration and call
\codestyle{AddTransportClass(FooTransport)} in either the \texttt{TransportStaticInit} function within libscopehal or
the \codestyle{PluginInit} function of a plugin, as appropriate.

The special class \codestyle{SCPINullTransport} serves as a \texttt{/dev/null} equivalent: it discards anything written
to it, and never returns read data. It is primarily intended to be used by the ``demo" driver, which does not connect
to a real instrument.

While it is in principle possible to create a driver class that talks directly to a device via e.g. a USB API and
bypasses the transport model, this is strongly discouraged for user experience and flexibility reasons. Most drivers
for such devices (for example the Digilent and Pico drivers) instead consist of two components: a bridge server that
converts the instrument API to SCPI commands on one socket and a raw sample data on a second socket, and a
libscopehal-side driver that converts this to the relevant instrument API.

\section{Oscilloscopes}
\label{sec:oscilloscopes}

An Oscilloscope is an instance of a class derived from \codestyle{Oscilloscope}, which represents a device for
acquiring sampled digital data. All actual oscilloscopes use this API, as do some other instruments such as spectrum
analyzers. Most oscilloscope driver classes derive from \codestyle{SCPIOscilloscope} rather than directly from
\codestyle{Oscilloscope}, as they use SCPI to communicate with the hardware.

An oscilloscope may have zero or more \hyperref[sec:channels]{channels}. \footnote{All currently extant implementations
have at least one channel, however it is plausible that a zero-channel instrument might exist in the future (for
example, some sort of external trigger controller that exposes the same trigger API as a conventional oscilloscope) so
the API allows for this.}

At any given time, an oscilloscope has exactly one \hyperref[sec:triggers]{trigger} associated with it. A trigger has
inputs and properties just like a \hyperref[sec:filters]{filter}, since both are derived from
\codestyle{FlowGraphNode}. Most triggers take at least one input, however zero-input triggers are possible (for
example, triggering on AC mains zero crossings).

Every oscilloscope driver class must contain a public static method \codestyle{GetDriverNameInternal()}, which returns a
\codestyle{std::string} containing a short, human readable name for the driver (for example ``agilent" or ``pico"). By
convention, the driver name should consist of lowercase letters and numbers only - no spaces, punctuation, or capital
letters.

Just like transports, every oscilloscope driver class must be registered in the dynamic creation table by invoking
\codestyle{OSCILLOSCOPE\_INITPROC(MyOscilloscope)} in the class declaration and
\codestyle{AddDriverClass(MyOscilloscope)} in \texttt{DriverStaticInit} or \texttt{PluginInit}.

\section{Channels}
\label{sec:channels}

A channel is an instance of a class derived from \codestyle{OscilloscopeChannel}, which represents a single source of
data and associated controls. A channel may be associated with an \hyperref[sec:oscilloscopes]{oscilloscope}, or it may
be a \hyperref[sec:filters]{filter} which is not associated with any particular physical instrument.

Channels of an oscilloscope generally map 1:1 to analog front ends. Most commonly they are also 1:1 with instrument front
panel connectors, however there are some notable exceptions. Some high end oscilloscopes (such as the Teledyne LeCroy
WaveMaster family) have multiple inputs with a multiplexer feeding a single front end; this ensemble is considered to
be a single channel by libscopehal. Network analyzers have separate channels for receive and reflected power, for
example $S_{11}$ and $S_{12}$ of a VNA are measured at the same physical port on the instrument but separate channels
in libscopehal.

A channel normally has one or more output \hyperref[sec:streams]{streams}, however in some less common situations (such
as dedicated trigger inputs) there may be zero streams.

Channels are reference counted: when at least one filter or waveform view is consuming the output of a channel it will
be automatically enabled. When the last user of a channel is removed, the channel will be disabled and, if a filter,
deleted.

Various properties can be configured on channels, such as gain/offset and bandwidth limiters. Depending on whether the
channel is a filter or not, or what kind of oscillocope it is connected to, not all of these settings may be available.

\section{Streams}
\label{sec:streams}

A stream is an output from a \hyperref[sec:channels]{channel}. Most channels of physical oscilloscopes have only a
single stream, however some have multiple (for example I and Q from a realtime spectrum analyzer, or magnitude and
angle from a VNA). Many filters have multiple output streams, for example each channel of an imported WAV file is a
separate stream of the import filter.

All streams of a channel must have the same X axis unit, however they may have independent Y axis units.

The set of streams provided by a filter may change at run time, most commonly if an import filter is pointed to a new
file. When a filter changes its set of output streams, it must emit the \codestyle{m\_outputsChangedSignal} signal so
that other code can handle the change appropriately.

\section{Triggers}
\label{sec:triggers}

TODO: write this section

\section{Waveforms}
\label{sec:waveforms}

A waveform is a class derived from \codestyle{WaveformBase} which stores a vector of sampled data. The
\codestyle{AnalogWaveform} and \codestyle{DigitalWaveform} classes store 32-bit floating point and Boolean data
respectively. Additional waveform classes are defined by many protocol decodes to store data of arbitrary class type.

The units for X and Y axis are not specified in the waveform, but are properties of the channel / stream that the
waveform came from. Most commonly, for analog oscilloscope waveforms, the X axis unit is femtoseconds and the Y axis
unit is volts - but other units may be encountered, for example the output of a FFT has X axis units in Hz and Y axis
in dBm.\footnote{Some variables and methods throughout the project (especially in older code) use ``time" or
``voltage" terminology to refer to the current X or Y axis units. This will likely be changed through refactoring over
the long term.}

Waveforms store timestamp / header metadata as well as three vectors of data:

\begin{itemize}
\item \codestyle{m\_offsets}: start time of each sample
\item \codestyle{m\_durations}: length of each sample
\item \codestyle{m\_samples}: actual sample data
\end{itemize}

All three vectors must always be the same length. (The struct-of-arrays memory format allows for better cache locality
and is more SIMD-friendly than an array-of-structs format.)

Sample offsets and durations are measured in time base units (defined by \codestyle{m\_timescale}). This is commonly
the sample rate of the ADC or logic analyzer that acquired the data, however for upsampled or interpolated data smaller
time scale values - as low as 1 - may be used. A static offset, the ``trigger phase" (\codestyle{m\_triggerPhase}),
measured in raw X axis units and not scaled by \codestyle{m\_timescale}, is added to the timestamp of every signal
after scaling by the time base unit. This is commonly used to apply a sub-sample offset to a waveform for trigger
interpolation or de-skewing.

The final timestamp of sample \emph{i}, in X axis units, is thus \codestyle{m\_offsets[i]}*\codestyle{m\_timescale} +
\codestyle{m\_triggerPhase}.

Note that the offset/duration allows samples to have arbitrary length and spacing; i.e. waveforms are inherently
sparse. This is necessary to support protocol events, irregularly sampled data, etc. Sample timestamps must increase
monotonically: sample \emph{i+1} must start at or after the end of sample \emph{i}.

The majority of waveforms (such as those coming directly off an oscilloscope) will be uniformly sampled, which renders
the sparse storage format inefficient. A waveform of N samples which has a duration of 1 for every sample, and offsets
ranging from 0 to \emph{N-1}, is considered to be ``dense packed" and should have the \codestyle{m\_densePacked} flag
set to enable various processing optimizations. The dense pack flag must NOT be set on a waveform which does not meet
these criteria as this can lead to incorrect output.

Filters presented with input marked as dense packed are free to ignore the timestamp and duration flags at their input.
Filters generating densely packed output should set the dense pack flag, however they must still fill the timestamp and
duration vectors for use by filters which do not have an optimized special case for dense packed inputs.

\section{Filters}
\label{sec:filters}

TODO: write this section

\section{Plugins}

A plugin is a shared library which may contain transports, drivers, filters, and export wizards. All of these must be
registered in a function called \codestyle{PluginInit} exported with extern "C" linkage.

Plugins are automatically loaded at startup by glscopeclient, however standalone applications using libscopehal must
explicitly call \codestyle{InitializePlugins()} to load them.

\subsection{Linux}

On Linux, plugins are loaded from the following directories:

\begin{itemize}
\item \codestyle{/usr/lib/scopehal/plugins}
\item \codestyle{/usr/local/lib/scopehal/plugins}
\item \codestyle{~/.scopehal/plugins}
\item Executable directory, if not under \codestyle{/usr}
\end{itemize}

\subsection{Windows}

On Windows, plugins are loaded from the following directories:

\begin{itemize}
\item \codestyle{(Executable directory) \textbackslash plugins}
\end{itemize}
